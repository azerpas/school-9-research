%===============================================================================
% $Id: ifacconf.tex 19 2011-10-27 09:32:13Z jpuente $  
% Template for IFAC meeting papers
% Copyright (c) 2007-2008 International Federation of Automatic Control
%===============================================================================
\documentclass{ifacconf}

\usepackage{graphicx}      % include this line if your document contains figures
\usepackage{natbib}        % required for bibliography
%===============================================================================
\begin{document}
\begin{frontmatter}

\title{Transiting from energy consuming traditional blockchains to sustainable ones\thanksref{footnoteinfo}} 
% Title, preferably not more than 10 words.

\thanks[footnoteinfo]{Sponsor and financial support acknowledgment
goes here. Paper titles should be written in uppercase and lowercase
letters, not all uppercase.}

\author[First]{Anthony Manikhouth}

\address[First]{EFREI Paris, 30 Av. de la République, 94800 Villejuif (e-mail: anthony.manikhouth@efrei.net).}

\begin{abstract}                % Abstract of not more than 250 words.
These instructions give you guidelines for preparing papers for IFAC
technical meetings. Please use this document as a template to prepare
your manuscript. For submission guidelines, follow instructions on
paper submission system as well as the event website.
\end{abstract}

\begin{keyword}
Decentralized and distributed control, Sustainability, Migration, Optimization and control of large-scale network systems, Blockchain
\end{keyword}

\end{frontmatter}
%===============================================================================

\section{Introduction}
Cryptocurrencies have become tremendously famous nowadays. Whether the appeal is technology, gains, or freedom, Bitcoin, Ethereum and other cryptocurrencies have become a subject of controversies and regulations in a lot of countries. Their environmental impact has been the center of criticism for a few years already and while a lot of studies reflect that \textit{Proof-of-Work} algorithms consume tons of energy, these cryptocurrencies are still at the top of any trading platform. 

However, legions of new blockchains are created everyday with the same goal,  become more energy efficient, scalable, and interoperable. Few of them make it to the stage of being actually used. While energy prices are rising up to unmatched levels, mining farms are still operating at full speed and authorities, in their need of saving up energy power, are starting to investigate this activity.  

Whereas Ethereum started to transiting to a new consensus algorithm: \textit{Proof-of-Stake}, and some other new blockchains like Solana or Near are flowing into the market with strong environmental values, we can wonder how blockchains will shift to some more environmental-friendly technology and sustain in the long term.


\section{Bitcoin and the Proof-of-Work}

As we said earlier, Bitcoin remains the most popular \textit{Proof-of-Work} blockchain, if not the most popular blockchain of all. 

\subsection{Bitcoin}

Bitcoin find its roots in 2009 when Satoshi Nakamoto released the first open-source version. This digital currency use a ledger called blockchain to store all of the transactions which are represented by cryptographic hashes. To verify the integrity of these transactions, the network is composed of nodes which all can verify the validity of each new block, with the latest containing multiple transactions.

\subsection{Proof-of-Work and Decentralization}

One of the strengths of \textit{Bitcoin} and its \textit{Proof-of-Work} system is its resistance to attacks. 
\begin{pf}  
You would need to take possession of 51\% of the active nodes to corrupt the network, which at the time of this writing accounts for at least 130 millions terrahash/s. To achieve this, you would need 1,3 millions Antminer S19 Pro (the current best ASIC Bitcoin Miner on the market) with each producing 100 TH/s, at a whooping total price of 19 billions dollars. And this is excluding the price of the electricity required to run the miners, with each consuming 3250W, the individual cost would be 10\$/day each.
\end{pf}

As observed, the energy required to attack the Bitcoin network is enormous.
This energy consumption is at least making Bitcoin the most transparent and decentralized currency in the world. And that is why, in its almost 14 years of existence, Bitcoin hasn't shifted to another form of consensus. The community around this blockchain is concerned by this decentralization that there haven't been any progress made towards some other consensus which could possibly fragilize Bitcoin.

 The estimated total global energy usage for Bitcoin is 168 billion kilowatt-hours per year at highest, which is between 0.4 to 0.9\% of the total global annual electricity usage in the world or more than 50\% of the energy used by all of the datacenters in the world, see \cite{TowGreen:21}.

However, in recent years, there has been a pursuit of a more sustainable computer science, especially concerning the green coding and the effects of the results. Concerns about the \textit{Proof-of-Work} consensus have been \textit{Ethereum}.

\section{Ethereum and the Proof-of-Stake}

Launched in 2015, Ethereum is a project aiming to bring programming to the blockchain. Today it is the blockchain that has the most value locked in dollars, which implies that it is also the most used. Until weeks ago, the \textit{Proof-of-Work} consensus was used to ensure the security of the blockchain. What has change?

\subsection{Smart contracts}
Smart contracts are programs that are able to store states and interact with it on the blockchain. Ethereum was the first to made it possible through and also the first to create programming languages, Solidity, Vyper, that can interact with EVM, the Ethereum Virtual Machine. While Solidity is very strong and supports most of the functionnalities that classic programming languages support, it also brings a lot of security concerns. At the time of writing this article, Ethereum TVL (\textit{Total Locked Value}) is 25 billion dollars with a maximum reached in Nov. 2021 of 100 billion dollars, see \citep[reference.][]{DefiLlama:22}. Smart contracts are good solutions to improve trust in computer programs as they are public on the blockchain, visible to anyone and audit-able by anyone. Plus, the computing power required by Ethereum and its \textit{Proof-of-Work} consensus algorithm to run these smart contracts, is another barrier to prevent malicious actors from spamming the network.

As we said earlier on, \textit{Proof-of-Work} consensus was used until very recently on Ethereum blockchain. Ensuring that the only way to hack your way around the network was to overpower the network, and as we saw we Bitcoin, it proves itself very tricky. 

But in 2022, Ethereum chose to drift from \textit{Proof-of-Work} to a  well-known consensus algorithm: \textit{Proof-of-Stake}.

\subsection{Proof-of-Stake}

\textit{Proof-of-Stake (PoS)} is a consensus algorithm that relies on the number of coins that each participant of the network holds and stakes. For example, on the Ethereum blockchain, if you possess some ethers (the official coin of the blockchain), stake them indefinitely, and get selected based on your stakes, you will get rewarded by some more ethers for securing the network. So instead of using a big wasting energy consumer like \textit{Proof-of-Work}, you could use \textit{PoS} which use a pseudo-random algorithm to determine the selected staker that will verify the transaction. The probability that a node is selected to verify a transaction is defined as follow:
\begin{equation} \label{eq:sample}
{p_{i}} = {{s_{i}}\over {\sum_{j=1}^{N} s_{j}}}.
\end{equation}
\begin{itemize}
    \item[$p_{i}$] being the probability for a node to be selected
    \item[i] being the node
    \item[N] the number of participants in the network
    \item[$s_{i}$] the stake of the node \textit{i}
\end{itemize}


% \begin{itemize}
%     \item Salut
% \end{itemize}

% \begin{equation} \label{eq:sample}
% {{\partial F}\over {\partial t}} = D{{\partial^2 F}\over {\partial x^2}}.
% \end{equation}

% \subsubsection{Example.} This equation goes far beyond the
% celebrated theorem ascribed to the great Pythagoras by his followers.

% \begin{thm}   % use the thm environment for theorems
% The square of the length of the hypotenuse of a right triangle equals
% the sum of the squares of the lengths of the other two sides.
% \end{thm}

% \begin{pf}    % and the pf environment for proofs
% The square of the length of the hypotenuse of a right triangle equals the sum of the squares 
% of the lengths of the other two sides.
% \end{pf}

% %% There are a number of predefined theorem-like environments in
% %% ifacconf.cls:
% %%
% %% \begin{thm} ... \end{thm}            % Theorem
% %% \begin{lem} ... \end{lem}            % Lemma
% %% \begin{claim} ... \end{claim}        % Claim
% %% \begin{conj} ... \end{conj}          % Conjecture
% %% \begin{cor} ... \end{cor}            % Corollary
% %% \begin{fact} ... \end{fact}          % Fact
% %% \begin{hypo} ... \end{hypo}          % Hypothesis
% %% \begin{prop} ... \end{prop}          % Proposition
% %% \begin{crit} ... \end{crit}          % Criterion

% Of course LaTeX manages equations through built-in macros. You may
% wish to use the \textit{amstex} package for enhanced math
% capabilities.

% \subsection{Figures}

% To insert figures, use the \textit{graphicx} package. Although other
% graphics packages can also be used, \textit{graphicx} is simpler to
% use. See  Fig.~\ref{fig:bifurcation} for an example.

% \begin{figure}
% \begin{center}
% \includegraphics[width=8.4cm]{bifurcation}    % The printed column width is 8.4 cm.
% \caption{Bifurcation: Plot of local maxima of $x$ with damping $a$ decreasing} 
% \label{fig:bifurcation}
% \end{center}
% \end{figure}

% Figures must be centered, and have a caption at the bottom. 

% \subsection{Tables}
% Tables must be centered and have a caption above them, numbered with
% Arabic numerals. See table~\ref{tb:margins} for an example.

% \begin{table}[hb]
% \begin{center}
% \caption{Margin settings}\label{tb:margins}
% \begin{tabular}{cccc}
% Page & Top & Bottom & Left/Right \\\hline
% First & 3.5 & 2.5 & 1.5 \\
% Rest & 2.5 & 2.5 & 1.5 \\ \hline
% \end{tabular}
% \end{center}
% \end{table}

% \subsection{Final Stage}

% Authors are expected to mind the margins diligently.  Papers need to
% be stamped with event data and paginated for inclusion in the
% proceedings. If your manuscript bleeds into margins, you will be
% required to resubmit and delay the proceedings preparation in the
% process.

% \subsubsection{Page margins.} See table~\ref{tb:margins} for the
% page margins specification. All dimensions are in \emph{centimeters}.


% \subsection{PDF Creation}

% All fonts must be embedded/subsetted in the PDF file. Use one of the
% following tools to produce a good quality PDF file:

% \subsubsection{PDFLaTeX} is a special version of LaTeX by Han The
% Thanh which produces PDF output directly using Type-1 fonts instead of
% the standard \textit{dvi} file. It accepts figures in JPEG, PNG, and PDF
% formats, but not PostScript. Encapsulated PostScript figures can be
% converted to PDF with the \textit{epstopdf} tool or with Adobe Acrobat
% Distiller.

% \subsubsection{Generating PDF from PostScript} is the classical way of
% producing PDF files from LaTeX. The steps are:

% \begin{enumerate}
%   \item Produce a \textit{dvi} file by running \textit{latex} twice.
%   \item Produce a PostScript (\textit{ps}) file with \textit{dvips}.
%   \item Produce a PDF file with \textit{ps2pdf} or Adobe Acrobat
%   Distiller.
% \end{enumerate}

% \subsection{Copyright Form}

% IFAC will put in place an electronic copyright transfer system in due
% course. Please \emph{do not} send copyright forms by mail or fax. More
% information on this will be made available on IFAC website.


% \section{Units}

% Use SI as primary units. Other units may be used as secondary units
% (in parentheses). This applies to papers in data storage. For example,
% write ``$15\,\mathrm{Gb}/\mathrm{cm}^2$ ($100\,\mathrm{Gb}/\mathrm{in}^2$)''. 
% An exception is when
% English units are used as identifiers in trade, such as ``3.5 in
% disk drive''. Avoid combining SI and other units, such as current in
% amperes and magnetic field in oersteds. This often leads to confusion
% because equations do not balance dimensionally. If you must use mixed
% units, clearly state the units for each quantity in an equation.  The
% SI unit for magnetic field strength $\mathbf{H}$ is $\mathrm{A}/\mathrm{m}$. However, if you wish to
% use units of $\mathrm{T}$, either refer to magnetic flux density $\mathbf{B}$ or
% magnetic field strength symbolized as $\mu_0\,\mathbf{H}$. Use the center dot to
% separate compound units, e.g., ``$\mathrm{A} \cdot \mathrm{m}^2$''.

% \section{Helpful Hints}

% \subsection{Figures and Tables}

% Figure axis labels are often a source of confusion. Use words rather
% than symbols. As an example, write the quantity ``Magnetization'', or
% ``Magnetization M'', not just ``M''. Put units in parentheses. Do not
% label axes only with units.  For example, write ``Magnetization
% ($\mathrm{A}/\mathrm{m}$)'' or ``Magnetization ($\mathrm{A} \mathrm{m}^{-1}$)'', not just
%  ``$\mathrm{A}/\mathrm{m}$''. Do not
% label axes with a ratio of quantities and units. For example, write
% ``Temperature ($\mathrm{K}$)'', not ``$\mbox{Temperature}/\mathrm{K}$''.

% Multipliers can be especially confusing. Write ``Magnetization
% ($\mathrm{kA}/\mathrm{m}$)'' or ``Magnetization ($10^3 \mathrm{A}/\mathrm{m}$)''. Do not write
% ``Magnetization $(\mathrm{A}/\mathrm{m}) \times 1000$'' because the reader would not know
% whether the axis label means $16000\,\mathrm{A}/\mathrm{m}$ or $0.016\,\mathrm{A}/\mathrm{m}$.

% \subsection{References}

% Use Harvard style references (see at the end of this document). With
% \LaTeX, you can process an external bibliography database 
% using \textit{bibtex},\footnote{In this case you will also need the \textit{ifacconf.bst}
% file, which is part of the \textit{ifaconf} package.}
% or insert it directly into the reference section. Footnotes should be avoided as
% far as possible.  Please note that the references at the end of this
% document are in the preferred referencing style. Papers that have not
% been published should be cited as ``unpublished''.  Capitalize only the
% first word in a paper title, except for proper nouns and element
% symbols.

% \subsection{Abbreviations and Acronyms}

% Define abbreviations and acronyms the first time they are used in the
% text, even after they have already been defined in the
% abstract. Abbreviations such as IFAC, SI, ac, and dc do not have to be
% defined. Abbreviations that incorporate periods should not have
% spaces: write ``C.N.R.S.'', not ``C. N. R. S.'' Do not use abbreviations
% in the title unless they are unavoidable (for example, ``IFAC'' in the
% title of this article).

% \subsection{Equations}

% Number equations consecutively with equation numbers in parentheses
% flush with the right margin, as in (\ref{eq:sample}).  To make your equations more
% compact, you may use the solidus ($/$), the $\exp$ function, or
% appropriate exponents. Use parentheses to avoid ambiguities in
% denominators. Punctuate equations when they are part of a sentence, as
% in

% \begin{equation} \label{eq:sample2}
% \begin{array}{ll}
% \int_0^{r_2} & F (r, \varphi ) dr d\varphi = [\sigma r_2 / (2 \mu_0 )] \\
% & \cdot \int_0^{\inf} exp(-\lambda |z_j - z_i |) \lambda^{-1} J_1 (\lambda  r_2 ) J_0 (\lambda r_i ) d\lambda 
% \end{array}
% \end{equation}

% Be sure that the symbols in your equation have been defined before the
% equation appears or immediately following. Italicize symbols ($T$
% might refer to temperature, but T is the unit tesla). Refer to
% ``(\ref{eq:sample})'', not ``Eq. (\ref{eq:sample})'' or ``equation
% (\ref{eq:sample})'', except at the beginning of a sentence: ``Equation
% (\ref{eq:sample}) is \ldots''.

% \subsection{Other Recommendations}

% Use one space after periods and colons. Hyphenate complex modifiers:
% ``zero-field-cooled magnetization''. Avoid dangling participles, such
% as, ``Using (1), the potential was calculated'' (it is not clear who or
% what used (1)). Write instead: ``The potential was calculated by using
% (1)'', or ``Using (1), we calculated the potential''.

% A parenthetical statement at the end of a sentence is punctuated
% outside of the closing parenthesis (like this). (A parenthetical
% sentence is punctuated within the parentheses.) Avoid contractions;
% for example, write ``do not'' instead of ``don' t''. The serial comma
% is preferred: ``A, B, and C'' instead of ``A, B and C''.


\section{Conclusion}

A conclusion section is not required. Although a conclusion may review
the main points of the paper, do not replicate the abstract as the
conclusion. A conclusion might elaborate on the importance of the work
or suggest applications and extensions.

\begin{ack}
Place acknowledgments here.
\end{ack}

\bibliography{ifacconf}             % bib file to produce the bibliography
                                                     % with bibtex (preferred)
                                                   
%\begin{thebibliography}{xx}  % you can also add the bibliography by hand

%\bibitem[Abigael O.]{TowGreen:21}
%B.C. Abigael O., Amalia D., Constantinos M., Vasilios K.,
%\newblock Towards a Green Blockchain: A Review of Consensus Mechanisms and their Energy Consumption
%\newblock In A.F. Round, editor, \emph{Advances in Enzymology}, page
%  1. Smart Circular Economy/ Emerging Technology.

%\bibitem[Able et~al.(1954)Able, Tagg, and Rush]{AbTaRu:54}
%B.C. Able, R.A. Tagg, and M.~Rush.
%\newblock Enzyme-catalyzed cellular transanimations.
%\newblock In A.F. Round, editor, \emph{Advances in Enzymology}, volume~2, pages
%  125--247. Academic Press, New York, 3rd edition, 1954.

%\bibitem[Keohane(1958)]{Keo:58}
%R.~Keohane.
%\newblock \emph{Power and Interdependence: World Politics in Transitions}.
%\newblock Little, Brown \& Co., Boston, 1958.

%\bibitem[Powers(1985)]{Pow:85}
%T.~Powers.
%\newblock Is there a way out?
%\newblock \emph{Harpers}, pages 35--47, June 1985.

%\bibitem[Soukhanov(1992)]{Heritage:92}
%A.~H. Soukhanov, editor.
%\newblock \emph{{The American Heritage. Dictionary of the American Language}}.
%\newblock Houghton Mifflin Company, 1992.

%\end{thebibliography}

\appendix
\section{A summary of Latin grammar}    % Each appendix must have a short title.
\section{Some Latin vocabulary}              % Sections and subsections are supported  
                                                                         % in the appendices.
\end{document}
